\chapter{Introduction}
\pagenumbering{arabic}
\section{研究背景・目的}
現実社会における物流配送計画,通信ネットワークの設計,スケジューリング問題など,
我々の身の回りにある多くの課題は,数理的には「組合せ最適化問題」として定式化することができる.
一般にこれらの問題では要素間の関係性や制約条件が複雑に絡み合っており,
解の候補数は問題の規模(頂点数や変数など)に対して指数関数的に増大する.

計算複雑性理論の観点からは,これらの実用的な問題の多くはNP困難(NP-hard)と呼ばれるクラスに属することが知られている.
NP困難な問題に対しては,多項式時間で厳密な最適解を求めるアルゴリズムが存在しないと予想されており(P$\neq$NP予想),
線形計画法や分枝限定法といった厳密解法を用いたとしても,大規模な問題に対して現実的な時間内で解を得ることは極めて困難である.

このような背景から,厳密解を求めることが困難な大規模問題に対し現実的な計算時間内で「十分に良質な」近似解を得るための手法として,
メタヒューリスティクスが広く研究されている.
自然界のプロセスに着想を得た遺伝的アルゴリズム(Genetic Algorithm; GA)や,群知能に基づく粒子群最適化(Particle Swarm Optimization; PSO),
アントコロニー最適化(Ant Colony Optimization; ACO)などはその代表例であり,これらは大域的な探索能力を持つことから,多くの組合せ最適化問題において有効性である.

しかしながら,メタヒューリスティクスには共通して解決すべき課題が存在する.それは「局所最適解(Local Optima)」への停滞である.
局所最適解とは,近傍のどの解よりも優れているが,全体の中で最も優れている大域的最適解ではない解を指す.
探索の過程で一度この局所最適解に収束してしまうと,単純な近傍探索や交叉操作だけではそこから抜け出すことができず,探索が停滞してしまう.

従来の研究では,突然変異率の調整や多様性の維持メカニズムなどが提案されてきたが,計算困難性の高い問題(Hard Instances)に対しては,
これらだけでは不十分な場合が多い.探索が停滞した際に,現在の解構造を適度に破壊しつつ,新たな有望領域へと探索を誘導するための,より強力かつ適応的なメカニズムが必要とされている.
